\documentclass[a4paper]{article}
% Kodowanie latain 2
%\usepackage[latin2]{inputenc}
\usepackage[T1]{fontenc}
% Można też użyć UTF-8
\usepackage[utf8]{inputenc}

% Język
\usepackage[polish]{babel}
% \usepackage[english]{babel}

% Rózne przydatne paczki:
% - znaczki matematyczne
\usepackage{amsmath, amsfonts}
% - wcięcie na początku pierwszego akapitu
\usepackage{indentfirst}
% - komenda \url 
\usepackage{hyperref}
% - dołączanie obrazków
\usepackage{graphics}
% - szersza strona
\usepackage[nofoot,hdivide={2cm,*,2cm},vdivide={2cm,*,2cm}]{geometry}
\frenchspacing
% - brak numerów stron
\pagestyle{empty}
\usepackage{listings}


% dane autora
\author{Barbara Zięba, Dominik Gulczyński}
\title{Dokumentacja projektu Scrabble}
\date{\today}

% początek dokumentu
\begin{document}
\maketitle
\section{Wstęp}
Jako projekt wieńczący kurs Programowania obiektowego napisaliśmy
program umożliwiający przeprowadzenie rozgrywki w Scrabble na
komputerze.
\section{Instrukcja dla użytkownika}
\subsection{Początek gry}
Po uruchomieniu programu użytkownik proszony jest o podanie imion graczy. Gracze będą prowadzili rozgrywkę w podanej kolejności zaczynając od podanego jako pierwszy. Można podać imiona od jednego do czterech graczy, i to własnie ta liczba rozpocznie rozgrywkę.
\subsection{Ruch gracza}
Gracz musi w swoim ruchu wykonać jedną z następujących akcji:
\begin{enumerate}
\item[] \textbf{Położenie słowa.} Aby położyć słowo gracz powinieć kliknąć kratkę planszy, w której będzie się znajdować pierwsza litera danego słowa. Uwaga: Może to być litera już znajdująca się na planszy.
W nowym okienku należy wpisać jakie to będzie słowo (również litery już znajdujące się na planszy) i zaznaczyć czy ma zostać położone pionowo czy poziomo. Wszystkie tworzone przy ruchu słowa muszą być poprawne, a dokładane litery łączyć się z już ułożonymi. Jeśli ruch będzie poprawny odpowiednie litery gracza pojawią się na planszy, a on otrzyma nowe.
\item[] \textbf{Wymiana liter} ef
\item[] \textbf{Pass} ert
\end{enumerate}
Po kliknięciu Ok na komunikat "Kolej następnego gracza." pokażą się liery na ręku kolejnego gracza.
\subsection{Koniec gry}
\section{Analiza obiektowa}
\subsection{Zaimplementowane klasy}
\begin{enumerate}
\item[] \texttt{Game} co robi ta klasa
\end{enumerate}
\subsection{Diagram UML}
Diagram wygenerowany za pomocą InteliJ IDEA.
\subsection{Realizowana funkcjonalność}
\section{Wskazówki dla programistów}
Projekt jest otwarty na dalszą pracę nad nim. Jedną z opcji jego prostego rozbudowania jest umożliwienie gry w innych językach lub na innych zasadach.
\end{document}