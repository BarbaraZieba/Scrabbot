\documentclass[a4paper]{article}
% Kodowanie latain 2
%\usepackage[latin2]{inputenc}
\usepackage[T1]{fontenc}
% Można też użyć UTF-8
\usepackage[utf8]{inputenc}

% Język
\usepackage[polish]{babel}
% \usepackage[english]{babel}

% Rózne przydatne paczki:
% - znaczki matematyczne
\usepackage{amsmath, amsfonts}
% - wcięcie na początku pierwszego akapitu
\usepackage{indentfirst}
% - komenda \url 
\usepackage{hyperref}
% - dołączanie obrazków
\usepackage{graphics}
% - szersza strona
\usepackage[nofoot,hdivide={2cm,*,2cm},vdivide={2cm,*,2cm}]{geometry}
\frenchspacing
% - brak numerów stron
\pagestyle{empty}
\usepackage{listings}


% dane autora
\author{Barbara Zięba, Dominik Gulczyński}
\title{Dokumentacja projektu Scrabble}
\date{\today}

% początek dokumentu
\begin{document}
\maketitle
\section{Wstęp}
Jako projekt napisaliśmy program umożliwiający przeprowadzenie rozgrywki w Scrabble na komputerze. 
\section{Analiza obiektowa}
\subsection{Zaimplementowane klasy}
\begin{enumerate}
\item[] \texttt{Game} co robi ta klasa
\end{enumerate}
\subsection{Diagram UML}
Diagram wygenerowany za pomocą InteliJ IDEA.
\section{Instrukcja dla użytkownika}
\end{document}