\documentclass[a4paper]{article}
% Kodowanie latain 2
%\usepackage[latin2]{inputenc}
\usepackage[T1]{fontenc}
% Można też użyć UTF-8
\usepackage[utf8]{inputenc}

% Język
\usepackage[polish]{babel}
% \usepackage[english]{babel}

% Rózne przydatne paczki:
% - znaczki matematyczne
\usepackage{amsmath, amsfonts}
% - wcięcie na początku pierwszego akapitu
\usepackage{indentfirst}
% - komenda \url 
\usepackage{hyperref}
% - dołączanie obrazków
\usepackage{graphics}
% - szersza strona
\usepackage[nofoot,hdivide={2cm,*,2cm},vdivide={2cm,*,2cm}]{geometry}
\frenchspacing
% - brak numerów stron
\pagestyle{empty}
\usepackage{listings}


% dane autora
\author{Barbara Zięba, Dominik Gulczyński}
\title{Dokumentacja projektu Scrabble}
\date{\today}

% początek dokumentu
\begin{document}
\maketitle
\section{Wstęp}
Jako projekt wieńczący kurs Programowania obiektowego napisaliśmy
program umożliwiający przeprowadzenie rozgrywki w Scrabble na
komputerze.
\section{Instrukcja dla użytkownika}
\subsection{Początek gry}
Po uruchomieniu programu użytkownik proszony jest o podanie imion graczy. Gracze będą prowadzili rozgrywkę w podanej kolejności zaczynając od podanego jako pierwszy. Można podać imiona od jednego do czterech graczy, i to własnie ta liczba rozpocznie rozgrywkę.
\subsection{Ruch gracza}
Gracz musi w swoim ruchu wykonać jedną z następujących akcji:
\begin{enumerate}
\item[] \textbf{Położenie słowa.} Aby położyć słowo gracz powinieć kliknąć kratkę planszy, w której będzie się znajdować pierwsza litera układanego słowa. Uwaga: Może to być litera już znajdująca się na planszy.
W nowym okienku należy wpisać jakie to będzie słowo (również litery już znajdujące się na planszy) i zaznaczyć czy ma zostać położone pionowo (\textit{vertical}) czy poziomo (\textit{horizontal}). Wszystkie tworzone przy ruchu słowa muszą być poprawne, a dokładane litery łączyć się z już ułożonymi. Jeśli ruch będzie poprawny odpowiednie litery gracza pojawią się na planszy, a on otrzyma nowe.
\item[] \textbf{Wymiana liter} Gracz może w swoim ruchu wymienić od jednej do 7 (wszystkich) liter, które posiada. Za taki ruch nie otrzymuje punktów. Aby dokonać wymiany należy kliknąć przycisk \textit{Exchannge tiles} znajdujący się obok literek na ręku, a następnie wpisać bez spacji litery do wymiany.
\item[] \textbf{Pass} Po kliknięciu przycisku \textit{Pass} gracz opuszcza swoją kolejkę. Może wykonać takiruch, kiedy na przeykład nie może już dołożyć żadnej litery.
\end{enumerate}
Po ruchu litery gracza zostają schowane i pojawia się komunikat o ruchu następnego gracza. Należy przekazać mu komputer i kliknąć \textit{OK}.
\subsection{Koniec gry}
\subsection{Cofanie ruchu}
Gracze sami powinni zdecydować, jak chcą korzystać z tej opcji. Sugerujemy, żeby w razie pomyłki we wpisywaniu użytkownik mógł zgłosić (niezwłocznie), że chce cofnąć ostatni ruch. Wtedy następny gracz powinieć skorzystać z przycisku \textit{Revert last move} i dać kolejną szansę koledze.
\subsection{Szczegółowe zasady gry}
Program kopiuje zasady gry Scrabble. Aby poznać reguły liczenia punktów i inne szczegóły gry, można zajrzeć na stronę \url{http://www.scrabble.info.pl/pobierz/zasady.pdf}. Nasz program korzysta ze słownika wyrazów dopuszczalnych w grach dostępnego na stronie Słownika Języka Polskiego \url{https://sjp.pl/slownik/growy/}.
\section{Analiza obiektowa}
\subsection{Zaimplementowane klasy}
\subsubsection{Board} Jedna z głownych klas programu. Przechowuje informacje o stanie planszy: położonych literach, specjalnych polach i słowniku, który decyduje, które wyrazy są poprawne. Posiada dwie metody. Pierwsza, \texttt{placeWord} przyjmuje słowo, które chcemy położyć na planszy i jego pozycję. Jeśli jest to poprawny ruch zwraca parę listę płytek użytych przez gracza i liczbę punktów zdobytych za tej ruch. Kolejna, \texttt{value} służy do liczenia punktów zdobytych za ułożenie danego słowa.
\subsubsection{Game} Najważniejsza klasa informująca o przebiegu gry. Dziedziczy z klasy \texttt{Board}, a oprócz tego przechowuje listę graczy, listę ruchów (historię) i woreczek dostępnych literek.
\subsubsection{Rack} Po angielsku oznacza stojak na literki. Zawiera listę płytek i udostępnia metody, które pozwalają na aktualizację tej listy.
\subsubsection{Player} Klasa dziedzicząca po \texttt{Rack} i odpowiadająca za informacje na temat gracza. Zawiera jego imię i bieżącą liczbę punktów.
\subsubsection{Bag} Reprezentuje woreczek z płytkami, które zostały do wylosowania. Jej pole to lista płytek, a metody udostępniają operacje do wykonania na tej liście, np. wyjęcie losowej płytki.
\subsubsection{Move} Klasa ruchu. Przydatna do pamiętania historii gry i wyświetlania jej przebiegu.
\subsubsection{Tile} Reprezentuje płytkę na planszy. Tutaj mamy również statyczne metody \texttt{getValueOf} (mówi ile punktów jest warta dana litera) i \texttt{paintTile} (służy do rysowania płytek).
\subsubsection{Bonus} Odpowiada za specjalne pola planszy. Zawiera informacje informacje na temat bonusów powiększających liczbę punktów zdobywanych za ułożenie słowa. Posiada klasę wewnetrzną \texttt{Type} reprezentującą typy bonusów.
\subsubsection{Tree}
Drzewo wyrazów dopuszczonych do użytku w grze. Zostaje utworzone za pomocą pliku txt upublicznionego przez stronę www.sjp.pl. Zawiera wewnętrzną klasę \texttt{Node}. 
\subsubsection{Main}
Klasa, która uruchamia program.
\subsubsection{Rack}
\subsubsection{Rack}
\subsubsection{Rack}
\subsubsection{Rack}
\subsubsection{Rack}
\subsubsection{Rack}
\subsection{Diagram UML klas}
Diagram wygenerowany za pomocą InteliJ IDEA.
\section{Wskazówki dla programistów}
Projekt jest otwarty na dalszą pracę nad nim. Jedną z opcji jego prostego rozbudowania jest umożliwienie gry w innych językach lub na innych zasadach. Można również pokusić się o dodanie możliwości zmierzenia się w grze z komputerem.
\end{document}